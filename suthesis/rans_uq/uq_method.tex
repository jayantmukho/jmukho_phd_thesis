\section{Uncertainty and Error in RANS Simulations}

In order to use steady RANS simulations to design engineering products, it is essential to understand its strengths and shortcomings. The extensive use of mathematical models to accelerate these simulations makes it feasible to run multiple function evaluations at different test conditions. Concurrently, the inadequacy of the models in predicting real-world behavior of fluids limits its use to flow conditions where the models are valid. These restrictions have been found over the years through thorough verification and validation of CFD codes and turbulence models (TODO Citation). 

To have a productive discussion of uncertainties and errors in these simulations, precise language is required. For this purpose, definitions and guidelines established by the American Institute of Aeronautics and Astronautics (AIAA) in \cite{computational_fluid_dynamics_committee_guide_1998} are followed. In a non-academic setting, \textit{error} and \textit{uncertainty} are often used interchangeably. To avoid this ambiguity in the current discussion, uncertainty is defined as: 
\say{A potential deficiency in any phase or activity of the modeling process that is due to lack of knowledge} \cite{computational_fluid_dynamics_committee_guide_1998}. 

This is contrasted with the definition of error: 
\say{A recognizable deficiency in any phase or activity of modeling and simulation that is not due to lack of knowledge} \cite{computational_fluid_dynamics_committee_guide_1998}. Two key differences surface from these definitions: 

\begin{enumerate}
    \item \textbf{Potential vs. recognizable deficiency -} The effects of the uncertainty in some model may or may not cause a knowable deficiency in the resulting prediction. On the other hand, deficiencies introduced due to modeling errors are identifiable upon examination. 
    
    \item \textbf{Lack of knowledge -} Uncertainty is caused due to lack of knowledge of some physical aspect of what is being simulated. Errors exist even with complete knowledge and often have established practices and methods that can be used to reduce them.  
\end{enumerate}

To solidify these definitions with examples, consider the uncertainty introduced by turbulence models in contrast to discretization errors caused by insufficient grid resolution. These are two factors that are explored in depth in the coming sections. Turbulence models are required to solve the closure problem with the RANS equations. There are numerous models, and numerous variations built upon those models (TODO Citation). Their hyper-parameters are often calibrated to work well in certain flow conditions. This means deficiencies in the employed model may or may not present themselves, depending on the flow regime. There is a fundamental lack of knowledge that necessitates the development and use of so many different kinds of turbulence models. For these reasons, the deficiencies introduced by turbulence models are considered \textit{uncertainties}.

Solving non-linear PDEs, such as the RANS equations, requires the discretization of the continuous spatial and temporal domains into discrete ones. The discrete spatial domain is known as the grid or mesh. An insufficient number of grid points introduces discretization errors. As the number of grid points increases, i.e. the size of the grid spacing tends to zero, the discrete representation approaches the original continuous domain. Correspondingly, the discretization error approaches zero. Thus, discretization error is caused due to the grid quality and limits of computational resources, rather than a lack of knowledge. It can be quantified using grid convergence studies \cite{american_society_of_mechanical_engineers_standard_2009}. For these reasons, deficiencies due to insufficient discretization are considered \textit{errors}. A similar argument can be made for temporal discretization.

There are numerous sources of errors and uncertainties that can plague a CFD simulation. A sampling of these sources are listed in Table \ref{tab:errors_uncert}. Errors due to iterative convergence, programming, and usage are easily mitigated. Computer round-off error is inevitable yet inconsequential when compared to discretization error. This chapter focuses on turbulence modelling uncertainties and it's relationship with discretization errors. Section \ref{sec:equips_rans_uq} introduces the eigenspace perturbation methodology which provides a framework to estimate the uncertainties introduced by turbulence models. This methodology is implemented in the open-source CFD software, SU2 \cite{su2_aiaajournal}. Verification and validation of this methodology is presented in Section \ref{sec:VandV_rans_uq}. An extended example of the methodology's performance is presented in \ref{sec:crm_rans_uq}. 

\begin{table}[]
    \centering
    \def\arraystretch{1.2}
    \begin{tabular}{c|c}
         \textbf{Errors} &  \textbf{Uncertainties} \\ \hline
         Discretization & Turbulence modeling \\
         Iterative convergence & Simulation conditions \\
         Computer round-off & Complexity of physics \\
         Programming & \\
         Usage & \\
    \end{tabular}
    \caption{Sources of errors and uncertainties in CFD simulations. This is not an exhaustive list.}
    \label{tab:errors_uncert}
\end{table}