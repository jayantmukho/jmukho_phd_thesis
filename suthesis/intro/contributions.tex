\section{Contributions} \label{intro_contributions}

This thesis establishes a framework to perform virtual flight testing of an aircraft early in the design process.
While the focus is on aircraft design, the principles of multi-fidelity modeling and uncertainty quantification, and certification testing can be applied to any design problem that requires significant analysis. 

Starting with UQ, Chapter \ref{chap:rans_uq} delves into the eigenspace perturbation methodology to quantify epistemic uncertainties introduced by turbulence models into RANS CFD simulations.
The methodology is implemented in an open-source CFD solver, SU2, to enable its widespread use in the research community. 
It is validated on a bevy of test cases that range from commonly used benchmark flow conditions to those of specific aerospace interest. 
The relationship between turbulence modeling uncertainty and numerical discretization error, another common source of uncertainty in CFD simulations, is investigated.
The UQ methodology is applied to two aircraft, the NASA Common Research Model and the Generic T-tail Transport, to create aerodynamic databases with physics-informed uncertainties. 

Since CFD is not the only analysis technique used in the aircraft design process, Chapter \ref{chap:mf_gp} presents multi-fidelity Gaussian processes (GP) that are used to combine data from different information sources. 
Existing equations for multi-fidelity GP regression are extended to use noisy data when design sets are not nested.
Multi-fidelity aerodynamic databases are modeled using these equations.
AVL simulations are the lowest fidelity level, RANS CFD simulations are the middle-fidelity, and wind tunnel experimental data are the highest fidelity.
The benefits of multi-fidelity data fusion are elucidated using one-dimensional aerodynamic databases created for the NASA CRM aircraft. 
This is extended in Chapter \ref{chap:aero_db}, where the first comprehensive, multi-fidelity, multi-dimensional aerodynamics and controls databases representing a full-configuration aircraft's lateral and longitudinal dynamics are presented. 
These are created for the Generic T-tail Transport (GTT) aircraft.

With all aspects of an aircraft's performance defined, Chapter \ref{chap:cba} delves into the simulation of an FAA flight certification maneuver used to test commercial jets' air-worthiness.
The Monte Carlo method is used to propagate the uncertainties in the design analyses through the flight simulation.
The effect of the input uncertainty on the aircraft maneuver is analyzed, and a probability of succeeding/failing the certification test is explicitly quantified. 
The virtual flight testing is also performed using databases based on lower-fidelity data, representing what would be available at earlier stages in the design process.
By quantifying the likelihood of success/failure of a flight certification maneuver, potential problems in the aircraft design can be identified, and the risk of failure can be mitigated. 

These contributions are discussed in further detail in Chapter \ref{chap:conclusions}.
Potential avenues for future research are presented as well. 