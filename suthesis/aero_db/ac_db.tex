\section{Aerodynamics and Controls Databases}

An aircraft flies through a multitude of operating conditions during a mission. During takeoff and landing the aircraft flies through dense air in its high-lift configuration at low speeds and high angle of attack. While cruising, the aircraft flies steady and wings-level at high altitude with little controller input. Understanding an aircraft's predicted behavior in all these operating conditions is required for successful certification of the aircraft. 

This work builds large aerodynamics and controls databases by combining multiple sources of information and their associated uncertainties. The generic T-tail transport aircraft is used to demonstrate this capability and consequently all the work shown in this chapter will concern itself with this configuration.

The coefficients that are used to create the aerodynamic database are listed in Table \ref{tab:aero_db}. Force and moment coefficients are two-dimensional functions of angle of attack ($\alpha$) and sideslip angle ($\beta$). Stability derivatives are one-dimensional functions of $\alpha$. 
\begin{table}
    \renewcommand{\arraystretch}{1.2}
    \centering
    \begin{tabular}{ c|c|c|c } 
%  \hline
         Coefficient & Description & Input Variables & Group \\ 
         \hline
         $C_L$ & Coefficient of lift & $\alpha, \beta$  & \multirow{3}{5em}{Force coefficients}\\ 
         $C_D$ & Coefficient of drag & $\alpha, \beta$  \\
         $C_{SF}$ & Coefficient of side force & $\alpha, \beta$  \\ \hline
         $C_l$ & Coefficient of rolling moment & $\alpha, \beta$  & \multirow{3}{5em}{Moment coefficients} \\
         $C_m$ & Coefficient of pitching moment & $\alpha, \beta$  \\
         $C_n$ & Coefficient of yawing moment & $\alpha, \beta$  \\ \hline
         $C_{m_q}$ & Coefficient of pitching moment due to pitch rate & $\alpha$  & \multirow{5}{5em}{Stability derivatives}\\
         $C_{l_p}$ & Coefficient of rolling moment due to roll rate & $\alpha$ \\
         $C_{l_r}$ & Coefficient of rolling moment due to yaw rate & $\alpha$ \\
         $C_{n_p}$ & Coefficient of yawing moment due to roll rate & $\alpha$ \\
         $C_{n_r}$ & Coefficient of yawing moment due to yaw rate & $\alpha$
         \\
        %  \hline
    \end{tabular}
    \caption{List of aerodynamic coefficients that are used to make up the aerodynamic database}
    \label{tab:aero_db}
\end{table}

Similarly, the coefficients used to create the controls database are listed in Table \ref{tab:control_db}. Except for the case with elevator deflections, all coefficients are three-dimensional functions of $\alpha$, $\beta$, and control surface deflection ($\delta_*$). Since flaps change the baseline aerodynamics of the aircraft, its effect on force coefficients is included in the controls database. This is not the case for the other control surfaces, where only their effect on moment coefficients is used. 

\begin{table}
    \renewcommand{\arraystretch}{1.2}
    \centering
    \begin{tabular}{ c|c|c } 
%  \hline
         Control Surface & Coefficients Used & Input Variables \\ 
         \hline
         Ailerons & $C_l, C_m, C_n$ with deflected ailerons & $\alpha, \beta, \delta_a$  \\
         Elevator & $C_l, C_m, C_n$ with deflected elevator & $\alpha, \delta_e$  \\
         Rudder & $C_l, C_m, C_n$ with deflected rudder & $\alpha, \beta, \delta_r$  \\
         Flaps & $C_L, C_D, C_{SF}, C_l, C_m, C_n$ with deflected flaps & $\alpha, \beta, \delta_f$  \\
         Spoilers & $C_l, C_m, C_n$ with deflected spoilers & $\alpha, \beta, \delta_s$  \\
        %  \hline
    \end{tabular}
    \caption{List of controls surface coefficients that are used to make up the controls database}
    \label{tab:control_db}
\end{table}

It is important to note that true control derivatives, such as $C_{l_{\delta_a}}$, aren't calculated. Calculating the derivative using finite differencing is susceptible to numerical issues. Rather the change in the moment coefficient due to the deflected control surface is used. For example: 

\begin{equation}\label{equ:control_derivative}
    C_{l_{\delta_a}}(\alpha,\beta,\delta_a) = C_l(\alpha,\beta,\delta_a) - C_l(\alpha,\beta,\delta_a=0)
\end{equation}

The discrete data for each coefficient, and its associated uncertainty, is used to train a Gaussian process. Individual multi-fidelity Gaussian processes are used to model each coefficient.