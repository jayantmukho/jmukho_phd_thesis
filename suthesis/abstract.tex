\abstract{Explicit quantification of uncertainty in engineering simulations is being increasingly used to inform robust and reliable design practices. In the aerospace industry, computationally-feasible analyses for design optimization purposes often introduce significant uncertainties due to deficiencies in the mathematical models employed. In this paper, we discuss two recent improvements in the quantification and combination of uncertainties from multiple sources that can help generate probabilistic aerodynamic databases for use in aerospace engineering problems.  We first discuss the eigenspace perturbation methodology to estimate model-form uncertainties stemming from inadequacies in the turbulence models used in Reynolds-Averaged Navier-Stokes Computational Fluid Dynamics (RANS CFD) simulations. We then present a multi-fidelity Gaussian Process framework that can incorporate noisy observations to generate integrated surrogate models that provide mean as well as variance information for Quantities of Interest (QoIs). The process noise is varied spatially across the domain and across fidelity levels. Both these methodologies are demonstrated through their application to a full-configuration aircraft example, the NASA Common Research Model (CRM) in transonic conditions. First, model-form uncertainties associated with RANS CFD simulations are estimated. Then, data from different sources is used to generate multi-fidelity probabilistic aerodynamic databases for the NASA CRM.  We discuss the transformative effect that affordable and early treatment of uncertainties can have in traditional aerospace engineering practices. The results for one- and two-dimensional multi-fidelity databases are presented and compared to those from a Gaussian Process regression performed on a single data source. 
}