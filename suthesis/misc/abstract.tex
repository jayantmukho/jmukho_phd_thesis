\prefacesection{Preface}
The meteoric rise in computational capability has increased reliance on simulation techniques to inform aircraft design.
However, aircraft airworthiness testing for flight certification remains firmly rooted in real-world experiments performed after manufacturing an aircraft prototype.
Leveraging multi-fidelity modeling and uncertainty quantification, this thesis presents a framework that creates a stochastic representation of the aircraft, uses it to simulate flight certification maneuvers, and determines the likelihood of successfully meeting the certification requirement. 

The aircraft design process involves numerous analysis tools that predict the aircraft's performance.
Each tool has some uncertainty associated with its predictions due to an inadequate representation of real-world physics.
The quantification of the uncertainties associated with a popular analysis tool, Computational Fluid Dynamics simulations solving the Reynolds-Averaged Navier-Stokes equations, is presented, and the methodology is validated.

The simulation predictions and their associated uncertainties are combined with data from other analysis tools to create stochastic aerodynamics and controls databases.
The databases describe the aircraft's behavior across its flight envelope and provide probability distributions for its predictions based on the underlying analysis tool used to create the data. 
These databases are represented using multi-fidelity Gaussian processes. They use additive and multiplicative corrections to model the differences between analysis tools of varying fidelity.
Databases are generated for two aircraft configurations, the National Aeronautics and Space Administration (NASA) Common Research Model (CRM) and the Generic T-tail Transport (GTT) aircraft.

Deterministic samples of the databases, which represent slightly different aircraft behavior while respecting the underlying uncertainties, are created using Monte Carlo sampling techniques. 
Each sample is run through a flight simulation representing a real-world airworthiness test performed by the Federal Aviation Administration (FAA).
These virtual flight tests are agglomerated to create distributions of the performance metrics for the maneuver. 
The posterior distributions of the performance metrics enable the explicit quantification of the probability that the aircraft succeeds/fails in performing the certification maneuver. 
While simulation technology requires significant improvement to match the accuracy and reliability of real-world flight testing, this work is an essential first step towards performing aspects of flight certification through simulation alone. 

Instead of immediately replacing real-world testing, virtual flight testing opens new avenues for improving the aircraft design process.
Simulating flight certification testing before building a full-size aircraft prototype mitigates the enormous costs of expensive redesigns late in the aircraft design process.
The explicit calculation of the failure rates provides design suggestions to ensure the aircraft can meet the certification requirement with a prescribed success rate.