\section{Multi-Fidelity Analysis} \label{intro_mf}

During the course of the typical aerospace design process, different kinds of performance analysis tools are used at different stages.
Low-fidelity computer simulations accept lower accuracy for faster computations.
They are useful at the very early stages of the design process when the aircraft's geometry is not well defined and is subject to significant change.
For example, AVL \cite{drela2008athena} solves the incompressible potential flow equations, taking mere seconds.
It is used to rapidly explore a large multi-dimensional design space, changing variables like the number of engines or wing placement \cite{botero2016suave,botero2019generative}.
These are often replaced by higher-fidelity techniques, such as RANS CFD simulations, as the design progresses, and more details of the design are finalized.
Experimental data, normally obtained through a costly wind-tunnel test, typically provides the most accurate representation of the phenomena analyzed.
It is obtained late in the design process as expensive prototypes of subsystems need to be manufactured.

Instead of discarding the low-fidelity simulation data when higher-fidelity data is available, there are methods to combine data from multiple fidelity levels to better represent the quantities of interest (QOI).
Multi-fidelity extensions to popular surrogate modeling techniques have been developed.
Polynomial chaos expansion (PCE) \cite{oladyshkin2012data,blatman2011adaptive}, which is a popular technique for sensitivity  analysis \cite{sudret2008global,crestaux2009polynomial}, can be used to combine multi-fidelity information by learning an additive correction on the low-fidelity data \cite{ng2012multifidelity, palar2018global}.
Gaussian processes (GP) \cite{krige1951statistical,matheron1963principles,rasmussen_gaussian_2006}, popular for their direct estimation of the error in its modeling, combines multiple information sources by learning additive and multiplicative corrections on low-fidelity data \cite{kennedy_predicting_2000}.
This correction of the lower-fidelity data $f_{i-1}$ to a higher-fidelity $f_i$ is represented as
\begin{equation}
    f_i(\mathbf{x}) = \rho_i f_{i-1}(\mathbf{x}) + \delta_i(\mathbf{x}),
\end{equation}
where $\rho_i$ is a constant multiplicative term, and $\delta_i(\mathbf{x})$ is the additive term.
These are sometimes referred to as bridge functions \cite{han_improving_2013}.
Note that multi-fidelity PCE neglect the multiplicative term $\rho_i(\mathbf{x})$ in their corrections.
One disadvantage of using GP vs. PCE is that the GP assumes a Gaussian distribution across the uncertainty intervals. 
PCE can handle different probability distributions, but the superior multi-fidelity handling and the direct error estimation of the GP regression make multi-fidelity GP the tool of choice.

Multi-fidelity GP regression has been used extensively in aerospace applications \cite{lam_surrogate_nodate,menhorn_multifideliy_nodate}, modeling oil reservoir production and pressures \cite{kennedy_predicting_2000}, hydrodynamic simulations \cite{le_gratiet_recursive_2014}, and biological tissue growth \cite{lee2020propagation}.
Numerous extensions to the framework have been proposed.
Ghoreishi et al. introduced the ability to have non-hierarchical information sources by relating each fidelity level to the highest fidelity, as opposed to each other \cite{ghoreishi_gaussian_2018}. 
This is unnecessary in the for this work as there is a well-defined hierarchy based on the physics that is modeled. 
Perdikaris et al. \cite{perdikaris_nonlinear_2017} significantly improved prediction capability by using deep Gaussian processes \cite{damianou2013deep}.
The high computational cost and the loss of a Gaussian posterior precludes its use in this effort. 
Han et al. \cite{han_improving_2013} proposed using gradient information to improve predictive capability.
Unfortunately, gradient information is only available for RANS CFD simulations through adjoint simulations \cite{jameson1988aerodynamic}, but not for the other information sources. 

This work does employ important improvements to the multi-fidelity GP process that were proposed by Gratiet \cite{gratiet_multi-fidelity_nodate}.
The new equations significantly reduced the GP's computational cost by splitting the data into individual information sources.
This results in the inversion of smaller matrices which is more computationally efficient. 
He also extends the multiplicative term to be a function of $\mathbf{x}$ instead of a constant.
Section \ref{sec:mf_modeling} presents these equations in detail and extends them for use with noisy data for design sets that are not nested. 