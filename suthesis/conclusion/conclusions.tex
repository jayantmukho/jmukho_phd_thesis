\chapter{Conclusions} \label{chap:conclusions}

\section{Summary}
The goal of this is to introduce a virtual flight testing framework that can be used at any point during the aircraft design process to determine the likelihood that the aircraft passes or fails a particular certification maneuver. 
This was done by contributing to the state-of-the-art in three disciplines, uncertainty quantification, multi-fidelity modeling, and certification by analysis. 

\subsection{Uncertainty Quantification}

In the realm of uncertainty quantification, the eigenspace perturbation methodology \cite{iaccarino_eig_pert} was implemented in SU2, validated on a suite of test cases, and applied to full configuration aircraft simulations.
The modular architecture of SU2, an open-source CFD code, allowed for an implementation focused on \textit{versatility}, such that the module can be used by experts and non-experts alike. 
The validation suite was chosen to include flow conditions that introduce model-form uncertainties in RANS simulations.
Flow features such as corner flows, shear planes, separation bubbles, and shocks were focused on. 
Additionally, it was necessary to have high-fidelity data available for the test case so that suitable comparisons could be made.
The methodology performed well across all flow conditions, predicting larger uncertainty bounds in areas where RANS simulations often struggle to capture the flow physics and, conversely, predicting smaller bounds where RANS simulations are usually accurate.
The high-fidelity data points often fell within the uncertainty estimates from the methodology, although this is not mathematically guaranteed. 

This is an important reminder that regardless of the model (in)adequacy of a simulation method, there may be unforeseen uncertainties and errors introduced in the predictions that can cause results to deviate from higher-fidelity information sources. 
Numerical discretization error arising from insufficient grid refinement is one such example that often plague CFD simulations. 
Its relationship with the turbulence modeling uncertainty was investigated using the NACA0012 airfoil and ONERA M6 wing test cases.
Given a minimum level of grid resolution, enough to capture the relevant flow features, further refinement did not significantly change the turbulence modeling uncertainty estimate.
Moreover, the uncertainty estimates were significantly larger than the discretization errors. 
This allows the use of coarser meshes to perform the perturbed simulations without losing much accuracy. 

The application of the methodology to the NASA Common Research Model (CRM) was presented in detail.
A parameter-sweep in angle of attack was performed.
Simulations at low angles of attack, where the turbulence model is able to accurately capture flow features, resulted in smaller uncertainty estimates when compared to those performed at high angles of attack, where flow separation is significant and the turbulence model is unable to provide accurate predictions
The predicted bounds did not encapsulate the experimental data due to well-known geometric discrepancies between the wind tunnel model and the model used for numerical simulations.
Post-processing of the individual perturbed simulations allowed visualization of the dominant flow features that contribute to the uncertainty estimates.
This provides a qualitative use-case to improve design decisions and future high-fidelity data gathering.

\subsection{Multi-Fidelity Modeling}

Multi-fidelity Gaussian processes were used to combine data from different information sources.
Improvements suggested by Gratiet \cite{gratiet_multi-fidelity_nodate}, which significantly reduce the computational cost of learning the Gaussian process, were implemented.
The computational savings compared to the original implementation \cite{kennedy_predicting_2000} were validated using analytical functions. 
The existing equations were extended so that uncertainty could be specified independently for each data point, even when the design sets for each fidelity level are not nested.
This new set of equations were used to create multi-fidelity probabilistic aerodynamic databases for a full aircraft configuration, the NASA Common Research Model (CRM).
The multi-fidelity fit was able to outperform the single-fidelity fit, particularly when high-fidelity data is sparse or localized to areas where low-fidelity functions are inaccurate.
This advantage is more pronounced when multi-dimensional databases are created.
As the number of high-fidelity data points is increased, the multi-fidelity and single-fidelity models start performing identically.

\subsection{Probabilistic Aerodynamics and Controls Databases}
A second aircraft configuration, the Generic T-tail Transport (GTT), is used to take the probabilistic aerodynamic databases a step further.
The wealth of experimental data from wind and water tunnels allowed for high-fidelity modeling of the aircraft's complete lateral and longitudinal dynamics, and the effects of its control surfaces.
The generation of multi-fidelity data and the associated uncertainties was discussed and visualizations of the resulting single-fidelity and multi-fidelity controls databases were compared.
Sparsity of the high-fidelity data in the control surface deflection dimension causes the error estimate from the GP regression to balloon between available data points when only high-fidelity data is used to create the databases.
With multi-fidelity fusion of information sources, the abundance of well-correlated low-fidelity AVL data improves the GP model and results in a smooth, uniformly low, error estimate across the control surface deflection dimension. 
This is a significant result as it is achieved without using any new high-fidelity evaluations, thereby reducing the uncertainty with negligible analysis cost. 

\subsection{Certification by Analysis}
A virtual flight testing framework is the culmination of all the work presented in the previous sections. 
An air-worthiness certification test maneuver formulated by the FAA \cite{romanowski_flight_2018} to ensure sufficient roll capability in engine-out scenarios, is used for this purpose.
The aerodynamics and controls databases represent all aspects of an aircraft's dynamics and can be used to perform flight simulations.
To this end, The Boeing Company's existing tools and expertise in flight simulation and control-law mixing \cite{control_law_patent} were leveraged.
A simplified 5 degree-of-freedom simulator that is run with an open-loop controls configuration is employed, details of which are presented. 
Success in meeting the certification test requirement is defined by the completion of the flight simulation without over-saturating any control surfaces.
The level of control surface saturation are referred to as pitch, roll, and yaw metrics for the elevator, ailerons and rudder, respectively. 

Each deterministic sample of the GP models that represent the aircraft databases is a valid instance of the aircraft. 
These instances have slight variations in their performance due to the uncertainty in the underlying data informing the models.
By taking multiple samples of the same aircraft and running each through the flight simulation, a Monte Carlo analysis of the certification test was performed. 
The likelihood of the aircraft failing the certification test is calculated by analyzing the percentage of failures.
One thousand samples was shown to be sufficient to get converged posterior mean and variance for the simulation metrics. 
The GTT aircraft passes the certification maneuver with $100\%$ certainty.
This is expected as the configuration is based on the CRJ 700, a currently certified aircraft. 

To make the maneuver more challenging, a set of modifications to the control surface functionality were made.
The modified certification test was carried out with databases that are built using information that represents the aircraft at various stages of the design process.
Early representations use low-fidelity AVL data that has large uncertainties associated with it.
Successive design stages introduce higher-fidelity information, in the form of RANS CFD simulation results and experimental data from wind and water tunnels, one at a time.
The incremental improvements in the performance predictions are propagated thorough to the certification test results. 
These are visualized using cumulative distribution functions of the simulation metrics. 
Benefits of fusing information sources are reflected in the reduced uncertainty in the metrics for simulations using 3-fidelity databases as compared to those using single-, high-fidelity information.

Finally, a fundamental shift in the design ethos is discussed.
The current standard is to use conservative factors of safety to account for possible uncertainties in the design analyses.
These factors are based on historical experience, which is not necessarily applicable to the new, more ambitious aircraft designs that are being undertaken.
With an explicit quantification of the failure rate, design decisions can be made to mitigate risks to a prescribed level.
These risks, and consequent design decisions, are directly related to the uncertainty in the analyses.
As simulation techniques continue to improve, so will the estimation of these failure rates and the design decision required to mitigate risks.
While this is an ambitious goal, it is a necessary one if Certification by Analysis of aircraft is to be achieved. 


% Data generated using AVL \cite{drela2008athena}, with SME provided uncertainties, formed the low-fidelity estimates for the aerodynamics and controls databases.
% RANS CFD simulations, with uncertainty estimates from the eigenspace perturbation methodology, formed the medium-fidelity level for the aerodynamics databases.
% Wind and water tunnel data with SME defined uncertainties formed the 



\section{Future Work}

Certification by Analysis is in the nascent stages of development.
While the goal - completing certain certification tests through simulation alone - is set, there is no defined path to get there, yet.
This work is an an initial exploration of one potential route, through quantification of uncertainties in simulations, their propagation through certification tests, and a statistical, rather than deterministic, analysis of results. 
There are numerous avenues for further research, some of which are outlined in this section. 

\subsection{Uncertainty Quantification}

For the purposed of this work, the uncertainties for the AVL simulations and the wind tunnel data was provided by subject matter experts (SME). 
These were based on historical experience in using these analysis techniques for aircraft design. 
Instead, more rigorous handling of uncertainties can be performed. 
Low-fidelity simulations like AVL are used in the conceptual design stage where large changes to design parameters are considered.
Parameter sweeps can be conducted to get a range of performance predictions. 
These ranges can be used represent the uncertainty in the design parameters at that stage of the process.
This is separate from the uncertainty due to the modeling simplifications made in the simulation itself, but it is a more rigorous methodology than basing it on SME.

For experimental data, methods for uncertainty quantification have been published \cite{coleman1995engineering} and widely adopted.
% While these uncertainties are more regularly quoted when providing experimental data, it is still not a universal practice. 
Using the raw sensor data for the wind and water tunnel experiments would allow for an explicit quantification of the systematic and precision uncertainties. 
Additional errors due tunnel specific phenomena, such as blockage or flow angularity, can be quantified with the help of the tunnel engineers.

A significant focus of this work was the quantification of uncertainties introduced in RANS CFD simulations due to the model-inadequacy of turbulence models. 
While the eigenspace perturbation methodology has been effective on a large variety of test cases, there are avenues for improvement. 
Currently the eigenspace is uniformly perturbed to the limits of physical realizability everywhere in the flow domain. 
This can lead to improbable flow conditions that can create an over-estimation of the uncertainty.
More stringent limits on the perturbation magnitude as proposed by \cite{mishra_perturbations_2019} can be employed.
Additionally, high fidelity DNS and LES data can be used to learn the ideal perturbation magnitude based on the relevant mean-flow features.
This would allow for spatial variation in the perturbation magnitude across the flow domain.

The principles behind the methodology are valid for any turbulence model that uses the eddy-viscosity hypothesis. 
The current work has used the SST turbulence model for all of the results.
Extensions of the methodology can be made to apply it to other popular turbulence models, such as the one-equation Spallart-Allmaras \cite{allmaras2012modifications}, for wider applications. 

\subsection{Multi-Fidelity Modeling}

Within the framework of multi-fidelity Gaussian processes, there have been many improvements that can be implemented into the existing framework to improve the aerodynamics and controls databases that are created.
Non-hierarchical information fusion \cite{lam_multifidelity_2015} could be used to combine multiple information sources that are of an unknown, or equal fidelity level. 
For example, the multiple wind tunnel campaigns for the GTT aircraft could be combined into a single surrogate model.
Gradient information, although expensive to generate, can be used to improve the quality of the surrogate models and reduce the error estimates \cite{han_improving_2013}.
This can be especially useful with CFD data where gradient calculations can be done through adjoint simulations. 
Similarly, non-linear information fusion \cite{perdikaris_nonlinear_2017} could further enhance the effect of low-fidelity data by learning more complex trends between the information sources.
Though care must be taken to ensure the computational cost of processing the GP equations doesn't balloon to intractable levels.

Outside of Gaussian processes, other multi-fidelity modeling techniques should be explored. 
In particular, multi-fidelity polynomial chaos expansions (PCE) \cite{ng_multifidelity_2014} preserve the essential features of GP models, such as deterministic sampling and providing mean and variance information.
PCE have the added advantage of being able to handle uncertainties with non-Gaussian distributions. 
This avenue would also provide insight into how much the results depend on the kind of multi-fidelity modeling technique that is used. 

\subsection{Certification by Analysis}

As mentioned earlier, certification by analysis (CbA) is in its nascent stages of development.
For simulations to be used as certification tools, they would have to provide predictions of equivalent or, more likely, higher quality than real-world flight testing. 
The \textit{CFD Vision 2030 Study} \cite{slotnick_cfd_nodate} provides an insight into the advancements in CFD simulations that are essential to achieve in the near future. 
More generally, low-fidelity techniques are not sufficient to reproduce experimental-level data. 
It is computationally expensive to create a controls database using CFD simulations, and AVL simulations are woefully inadequate in this regard.
They over-estimate the control effectiveness and carry significant uncertainties in their predictions.
This is evident from the results of the certification simulations in Section \ref{sec:cba_results} where the results from the AVL databases are not reflective of the results from high-fidelity databases. 

Some important trends could be learned by using historical aircraft design data to study the relationship between early design analyses and final flight certification performance.
The design process can be recreated, but with the added virtual flight testing analysis.
Similar to what was done for the GTT aircraft, databases can be built using only the information available at respective design stages. 
The resulting databases can be put through the virtual flight testing framework and the results analyzed. 
Design decisions made with this extra information can be compared to the actual design decisions taken.
This would provide a direct comparison to assess the benefits of the virtual flight testing in the design framework. 

Additional maneuvers to stress-test other parts of the aerodynamics and controls databases can be explored.
The maneuver chosen for this work relied heavily on aileron and rudder control authority. 
Similarly, air-worthiness testing is not the only part of the certification process that can be made virtual. 
Extensive structural testing is done with prototypes of wings and engines.
The relatively lower cost of structural analysis can provide a lower barrier-to-entry for CbA. 
Choosing tests based on the accuracy of low-fidelity methods in predicting performance, is key. 
If the low-fidelity methods are well developed and accurate in the area that is being tested, their adoption for CbA is far more likely.
The day that simulations replace real-world testing is far into the future. 
With these initial steps, a potential path to that day is being paved. 