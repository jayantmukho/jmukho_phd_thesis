\chapter{Conclusions} \label{sec:conclusions}

In this paper, two main contributions have been presented in the context of multi-fidelity UQ applications of interest to the aerospace engineering industry, and to the simulation-based engineering community: a method to quantify model-form uncertainties in RANS CFD simulations, and a multi-fidelity Gaussian Process framework that combines data sources of varying accuracy to provide better estimates of QoIs and their uncertainties. Both of these methodologies were showcased using a real-world probabilistic aerodynamic database for a full-configuration aircraft, the NASA Common Research Model. 

The RANS UQ methodology uses eigenspace perturbations of the modeled Reynolds' stress tensor to create flow fields that push against the physical realizability constraints of the stress tensor. This methodology provided interval estimates on the QoIs based on the model-form uncertainties associated with turbulence modeling. Simulations at low angles of attack, where the turbulence model is able to accurately capture flow features, had smaller bounds than those performed at high angles of attack, where flow separation is significant and the turbulence model is unable to provide accurate predictions. The predicted bounds did not encapsulate the experimental data due to well known geometric discrepancies between the wind tunnel model and the model used for numerical simulations that resulted from unaccounted aeroelastic twist.

Previous work in multi-fidelity Gaussian processes was advanced by introducing noise in the observations of the QoIs. The RANS CFD simulations and associated uncertainty predictions were combined with low-fidelity data from AVL simulations, and high-fidelity data from wind tunnel experiments to create multi-fidelity surrogate models for $C_L$, $C_D$, and $C_m$ for the NASA CRM. These multi-fidelity fits were more accurate than single-fidelity GP fits on just the high-fidelity data at a significantly lower cost, showing that lower-fidelity data that is well correlated with higher-fidelity data can serve to augment high-fidelity data to improve predictive capabilities.  This is especially true in multi-dimensional operating spaces where the quantity of interest would require the use of a prohibitive number of high-fidelity simulations or wind-tunnel data points.

Finally, we would like the current work to serve as an early (and, necessarily, incomplete) example of the direction of evolution that would be beneficial for future engineering simulations. As industry increases its reliance on numerical simulations for design and analysis, it is essential to understand and address the shortcomings of such simulations. Quantifying uncertainties introduced by these predictive methodologies is the first step in extending their value. It enables the use of a reliability-based design process instead of the traditional, deterministic design methodology that makes no allowance for modeling uncertainties. Additionally, combining predictions from all the analysis methods that are used in the design process improves their predictive capability with lower computational cost. 
