\section{Certification by Analysis} \label{intro_cba}

To fly a new aircraft design, it needs to go through rigorous air-worthiness testing to ensure it is safe and can perform predictably in a variety of different situations.
These tests are defined and carried out by the Federal Aviation Administration (FAA) \cite{romanowski_flight_2018} in the US.
They occur at the very end of the design process, once a functional full-scale aircraft is built. 
Failing a certification test at this stage would require a redesign that would be incredibly expensive. 
To prevent this from occurring, safety factors are employed to account for potential uncertainty and error in the design analyses.
The quantification of uncertainties in analyses can replace arbitrary safety factors with the explicitly calculated design margins required to account for the errors.
This is the cornerstone of reliability-based design processes \cite{reliability}.

Certification by analysis (CbA) purports that the passing of certification requirements can be done using purely simulation-based analyses. 
It is the logical conclusion to the current trend of increased reliance on computer simulations for design analysis.
CbA has garnered interest from the aerospace community with the maturing of simulation procedures.
To achieve this, simulations would have to be as accurate, if not more accurate, than what is possible with flight testing.  
Many required improvements to simulation capabilities \cite{slotnick_cfd_nodate} will take years to develop.
The effects of uncertainties on flight performance predictions and predicted performance in flight testing could be quantified in the interim.
This provides aircraft designers with a method to estimate the likelihood that a design will pass or fail a certification test.

\subsection{Uncertainty Propagation}

A common method for uncertainty propagation is to perform a Monte Carlo analysis.
This is a brute-force method of characterizing the effect of input uncertainties on an output quantity of interest (QoI) \cite{janssen2013monte,thompson1992monte}.
It involves running multiple deterministic calculations where input variables are randomly sampled from their respective probability distributions.
The results are analyzed to determine the effect of the variation in the input variables on the posterior probability distribution of an output quantity of interest (QoI). 
In this work's context, the deterministic calculations are the flight simulations, the input values are the aerodynamics and controls databases, and the QoI is the success/failure of the certification test. 

A significant advantage of using GP regression to represent the probabilistic aerodynamic databases is the ability to take deterministic samples of the database.
Each sample is a potential candidate aircraft that could explain the data used to create the databases. 
They represent aircraft that have a slightly different performance from each other. 
These differences respect the uncertainty associated with the underlying data.
Since multi-fidelity GP regression is used to model the databases, the input probability distributions are Gaussian. 
The flight simulation does not represent a linear transfer function.
Therefore the posterior probability distribution of the QoI is arbitrary. 
As such, cumulative density functions are used to describe the posterior probabilities. 

\subsection{Aircraft Maneuver Simulation}

Aerodynamic databases contain all the information needed to perform high-fidelity flight simulations.
With multi-fidelity aerodynamics and controls databases, a virtual representation of the GTT aircraft can fly through real-world flight certification maneuvers. 
With the help of industry experts at The Boeing Company, a maneuver was picked from the FAA's \textit{Flight Test Guide for Certification of Transport Category Airplanes} \cite{romanowski_flight_2018}.
Within Chapter \textit{5.3 Directional and Lateral Control} of the guide, the \textit{Lateral Control: Roll Capability \S 25.147(d)} maneuver was chosen.   
The testing procedure is paraphrased as: 
\begin{enumerate}
    \item The airplane starts in a trimmed state for steady straight flight at maximum takeoff speed.
    \item Establish a steady $30^\circ$ banked turn.
    \item Roll the airplane to a $30^\circ$ bank angle in the other direction.
    \item Aircraft must have sufficient roll authority to perform the $60^\circ$ change in bank angle in no more than $11$ seconds. 
    \item The aircraft must do this with one engine inoperative, specifically the one that makes this maneuver more difficult.

\end{enumerate}

This part of the work leans heavily on The Boeing Company's expertise in flight simulation and control law mixing.
Due to proprietary and patent restrictions, the exact implementation of the flight simulation code is unavailable, but enough information is provided to outline the simulations' overarching methodology and workflow.
This maneuver is simulated using a 5 degree of freedom (DOF) flight simulator that The Boeing Company uses in their design processes.
Details of the maneuver simulation process are discussed in Section \ref{sec:sim_procedure}.