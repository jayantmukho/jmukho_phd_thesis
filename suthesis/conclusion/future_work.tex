\section{Future Work}

Certification by Analysis is in the nascent stages of development.
While the goal - completing certain certification tests through simulation alone - is set, there is no defined path to get there, yet.
This work is an an initial exploration of one potential route, through quantification of uncertainties in simulations, their propagation through certification tests, and a statistical, rather than deterministic, analysis of results. 
There are numerous avenues for further research, some of which are outlined in this section. 

\subsection{Uncertainty Quantification}

For this work, subject matter experts (SME) provide the uncertainties for the AVL simulations and the wind tunnel data. 
These were based on historical experience using these analysis techniques for aircraft design. 
Instead, the uncertainties can be handled more rigorously.
Engineers use low-fidelity simulations like AVL to consider significant changes to design parameters in the conceptual design stage.
Conducting parameter sweeps results in a range of performance predictions that can represent the uncertainty in the design parameters at that stage of the process.
This uncertainty is different from the uncertainty due to the modeling simplifications made in the analysis tool itself, but it is a more rigorous methodology than basing uncertainties on SME.

For experimental data, methods for uncertainty quantification have been published \cite{coleman1995engineering} and widely adopted.
Using the raw sensor data for the wind and water tunnel experiments would allow for an explicit quantification of the systematic and precision uncertainties. 
Additional errors due to tunnel-specific phenomena, such as blockage or flow angularity, can be quantified with the help of the tunnel engineers.

A significant focus of this work was the quantification of uncertainties introduced in RANS CFD simulations due to the model inadequacy of turbulence models. 
While the eigenspace perturbation methodology has been effective on a large variety of test cases, there are avenues for improvement. 
Currently, the eigenspace is uniformly perturbed to the limits of physical realizability everywhere in the flow domain. 
This procedure can lead to improbable flow conditions that can overestimate the uncertainty.
More stringent limits on the perturbation magnitude proposed by \cite{mishra_perturbations_2019} can be employed.
Additionally, high fidelity DNS and LES data can be used to learn the ideal perturbation magnitude based on the relevant mean-flow features.
This improvement would allow for spatial variation in the perturbation magnitude across the flow domain.

The principles behind the methodology are valid for any turbulence model that uses the eddy-viscosity hypothesis. 
The current work has used the SST turbulence model for all of the results.
Extensions of the methodology can enable its application to other popular turbulence models, such as the one-equation Spallart-Allmaras \cite{allmaras2012modifications}. 

\subsection{Multi-Fidelity Modeling}

Within the framework of multi-fidelity Gaussian processes, many improvements can be implemented into the existing framework to improve the aerodynamics and controls databases created.
Non-hierarchical information fusion \cite{lam_multifidelity_2015} could be used to combine multiple information sources that are of an unknown or equal fidelity level. 
For example, a single surrogate model could combine the multiple wind tunnel campaigns for the GTT aircraft.
Gradient information, although expensive to generate, can be used to improve the quality of the surrogate models and reduce the error estimates \cite{han_improving_2013,yamazaki_derivative-enhanced_2013}.
This change can be beneficial with CFD data, where adjoint simulations can provide gradient information. 
Similarly, non-linear information fusion \cite{perdikaris_nonlinear_2017} could further enhance the advantages of using low-fidelity data by learning more complex trends between the information sources.
Though care must be taken to ensure the computational cost of processing the GP equations does not balloon to intractable levels.

Outside of Gaussian processes, other multi-fidelity modeling techniques should be explored. 
In particular, multi-fidelity polynomial chaos expansions (PCE) \cite{ng_multifidelity_2014} preserve the essential features of GP models, such as deterministic sampling and providing mean and variance information.
PCE have the added advantage of handling uncertainties with non-Gaussian distributions. 
This avenue would also provide insight into how much the results depend on the multi-fidelity modeling technique used. 

\subsection{Certification by Analysis}

As mentioned earlier, certification by analysis (CbA) is in its nascent stages of development.
For simulations to replace current certification tools, they would have to provide equivalent or, more likely, higher-quality predictions than real-world flight testing. 
The \textit{CFD Vision 2030 Study} \cite{slotnick_cfd_nodate} provides an insight into the advancements in CFD simulations that are essential to achieve in the near future. 
More generally, current low-fidelity techniques are not sufficient to reproduce experimental-level data. 
It is computationally expensive to create a controls database using CFD simulations, and AVL simulations are woefully inadequate in this regard.
They overestimate the control effectiveness and carry significant uncertainties in their predictions.
This shortcoming is evident from the results of the certification simulations in Section \ref{sec:cba_results} where the results from the AVL databases are not reflective of the results from high-fidelity databases. 

Analysis of historical aircraft design data can unearth critical trends between early design analysis and final flight certification performance. 
The design process can be recreated, but with the added virtual flight testing analysis.
Similar to the process for the GTT aircraft, databases can be built using only the information available at each design stage. 
Analyzing the virtual flight testing results with these low- and multi-fidelity databases, design decisions can be suggested and compared to the actual design decisions that were taken. 
This comparison provides a direct assessment of the benefits of virtual flight testing in the design framework. 

Additional maneuvers to stress-test other parts of the aerodynamics and controls databases can be explored.
The chosen maneuver for this work relied heavily on aileron and rudder control authority. 
Similarly, airworthiness testing is not the only part of the certification process that can be made virtual. 
Extensive structural testing is done with prototypes of wings and engines.
The relatively lower cost and relatively high accuracy of structural analysis can lower entry barriers for CbA. 
Choosing tests based on the accuracy of low-fidelity methods in predicting performance is critical. 
If the low-fidelity methods are well developed and accurate in the test-critical parts of the domain, their adoption for CbA is far more likely.
The day that simulations replace real-world testing is far into the future. 
This work paves a potential path to that day with these initial steps.