\section{Numerical Discretization Error vs. Turbulence Modeling Uncertainty}\label{sec:num_vs_turb_error}

Solving continuous equations on a discrete domain creates numerical discretization error.
In RANS CFD simulations, the continuous RANS equations that define fluid flow are solved on a discrete domain known as the mesh or grid.
Numerical discretization error can be reduced by increasing the number of discrete points in the domain.
As the discretization increases and the grid spacing tends towards $0$, the numerical error approaches $0$ as well. 
This is the basis for the "Grid Convergence Study" method for quantification of numerical discretization error \cite{american_society_of_mechanical_engineers_standard_2009}.

The methodology requires the use of a family of grids that are sequentially coarser but are generated using the same grid generation parameters.
This is most easily done with structured meshes where a very dense grid, which would result in minimal discretization error, is first generated.
Then each successive coarser grid level removes every other grid line in each direction.
In 2D and 3D computations, this results in the number of grid point reducing by a factor of 4 and 8 respectively, at each grid level.
It results in grids that are uniformly refined which isolates the effect of the discretization on the simulations. 

The eigenspace perturbation methodology discussed in \ref{sec:equips_rans_uq} only estimates uncertainties introduced by turbulence models.
These simulations are run on grids that introduce some degree of discretization error.
This section explores the relationship and the relative magnitudes of the two quantities.

\subsection{NACA0012 Airfoil} \label{sec:num_err_naca0012}

The same NACA0012 case presented in Section \ref{sec:equips_naca0012} is used here. 
This case is used in the 5th and 6th Drag Prediction Workshops as a verification study \cite{levy2013summary,roy2017summary}.
A grids that were used for those verification studies are used here. 
Details of the grids are shown in Table \ref{tab:naca0012_meshes}.

\begin{table}
    \renewcommand{\arraystretch}{1.2}
    \centering
    \begin{tabular}{ c|c|c|c|c } 
%  \hline
         Mesh Level & Nodes & Surface Nodes & Wall spacing & Approx. $y^+$  \\ 
         \hline
         L1 & $14,687,744$ & $4,097$ & $1.0\times10^{-7}~m$ & 0.025\\
         L2 & $3,673,856$ & $2049$ & $2.0\times10^-7~m$ & 0.05\\
         L3 & $919,424$ & $1,025$ & $4.0\times10^{-7}~m$ & 0.1\\
         L4 & $230,336$ & $513$ & $8.0\times10^{-7}~m$ & 0.2\\
         L5 & $57,824$ & $257$ & $1.6\times10^{-6}~m$ & 0.4\\
         L6 & $14,576$ & $129$ & $3.2\times10^{-6}~m$ & 0.8\\
         L7 & $3,584$ & $65$ & $6.4\times10^{-6}~m$ & 1.6\\
        
    \end{tabular}
    \caption{Details of the mesh family used to perform numerical discretization error quantification for the NACA0012 case.}
    \label{tab:naca0012_meshes}
\end{table}

The grid convergence for $C_L$ and $C_D$ are shown in \textcolor{red}{Figure \ref{}}.