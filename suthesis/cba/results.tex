\section{Results} \label{sec:cba_results}
With the aircraft maneuver and flight simulation process defined, results from running the GTT aircraft through the flight certification maneuver are discussed in this section.
There are two over-arching goals of this work: first, to provide a framework that brings flight testing earlier in the aircraft design process, and second, to create the most accurate flight simulation results while minimizing the cost of the underlying analyses. 
To this end, this section first focuses on comparing simulations that are run using purely high-fidelity experimental data, to those run using low-fidelity and multi-fidelity data that would be available early in the design process.
Then the cost-reduction aspect is explored by harnessing the benefits of multi-fidelity databases that use small subsets of the high-fidelity data. 
For all the results in this section, simulations are conducted with the right engine inoperative, and all modifications discussed in Section \ref{subsec:sim_mods} are applied.

utilize multi-fidelity GP models that are created using various combinations of low-fidelity AVL data, medium-fidelity CFD simulation data, and high-fidelity experimental wind tunnel data, to those that utilize single-fidelity GP models created using purely high-fidelity data. 

\subsection{Simulations using Single- vs. Multi-Fidelity Databases} \label{subsec:sf_vs_mf_cba}

