\section{Aircraft Configurations}

The methodologies presented in this thesis were developed and demonstrated using two aircraft configurations: the NASA Common Research Model (CRM), and the Generic T-tail Transport (GTT). 

\subsection{NASA Common Research Model}
The NASA CRM is a well-investigated full-configuration aircraft \cite{rivers_further_2012,rivers_experimental_2010} that was developed with the goal of creating a baseline geometry upon which numerous experimental and computational studies could be performed and compared.
It was originally created for the 4th Drag Prediction Workshop \cite{morrison20094th} and was used for the subsequent 5th and 6th editions of the workshop as well \cite{levy2013summary,morrison20166th,roy2017summary,tinoco2017summary}.
The wealth of experimental and computational data lends itself well for the purpose of showcasing the performance of the uncertainty quantification and multi-fidelity data fusion techniques developed in this thesis.

The model was designed by The Boeing Company and is based on the Boeing 777 aircraft with a modified wing. 
It is a conventional tube-wing configuration designed for a cruise Mach number of 0.85.
The NASA CRM was built to be modular such that additional components could be attached to the baseline geometry. 
Consequently, different configurations of the aircraft were tested in the wind tunnel:
\begin{enumerate}
    \item Baseline wing + fuselage model,
    \item Wing + fuselage + pylon and nacelle,
    \item Wing + fuselage + horizontal tail mounted at either $-2^\circ, 0^\circ,$ or $+2^\circ$.
\end{enumerate}
For the purpose of this work, the configuration with the wing + fuselage + horizontal tail mounted at $-2^\circ$ is used. 
An image of the 

\begin{figure}
\centering
\includegraphics[width=0.75\textwidth]{suthesis/images/}
\caption{Flow chart showing the implementation of EQUiPS within SU2 \label{fig:equips_overview}}
\end{figure}

\subsection{Generic T-tail Transport}