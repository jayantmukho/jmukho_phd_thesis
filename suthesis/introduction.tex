\chapter{Introduction} \label{intro}

The rapid improvement of computational capabilities in the recent past has increased the use of computer simulations to predict various physical phenomena. This, coupled with advances in the understanding of the underlying physics of these phenomena, has led to the development of computer simulations of varying sophistication and computational cost that can describe the relevant QoIs at different levels of fidelity. These simulations allow for the critical assessment of engineering designs significantly earlier in the design process than what was previously possible with purely experimental design campaigns. For instance, RANS CFD simulations have now become commonplace in aerospace engineering: full aircraft analyses utilizing meshes with 250-500 million cells are routinely carried out in engineering practice. The utility of these simulations is further extended by leveraging the discrete realizations of the computational simulations to create continuous representations of the functions of interest. These continuous representations are called surrogate models and can be rapidly sampled for data-intensive methods such as uncertainty quantification (UQ) or design optimization \cite{queipo2005surrogate,gorissen2010surrogate}. Building surrogate models that can accurately represent the physical phenomena of interest can greatly reduce resources required for analyses in the design process \cite{jeong2005efficient}.

An approach to build the most accurate surrogate models is to only use simulations of the highest fidelity. Unfortunately, high-fidelity function evaluations are computationally expensive and time consuming. Lower-fidelity approximations are often used in their stead as they are relatively inexpensive, but they do not model all the relevant physical phenomena correctly. In other words, simulations that model physical phenomena at different levels of complexity inject varying amounts of uncertainty in their predictions \cite{peherstorfer_survey_2018}. These uncertainties, which are introduced due to inadequacies in the physical models being solved, are referred to as model-form uncertainties. Quantifying and understanding these uncertainties is essential to develop designs that can reliably meet their performance requirements \cite{forrester_multi-fidelity_2007}. 

For a simple example, imagine trying to simulate the trajectory of a football to determine how much force is required to throw it a distance of 40 yards. In an effort to simplify the simulation, we make the assumption that the football is spherical in shape, instead of oblong. Making this simplification introduces model-form uncertainties due to the inadequacy of a sphere to capture the physics of an oblong football flying through the air. This will result in an incorrect calculation of the required force. But if the uncertainty introduced in the calculation by this simplifying assumption is known, it can be accounted for and the final force calculation can be adjusted to ensure that the football is thrown 40 yards with a prescribed rate of success. 

Similarly, if uncertainty information is included in performance predictions, the probability that a particular design will meet certain performance requirements can be calculated. This is the cornerstone of Reliability Based Design (RBD) processes \cite{reliability}, which aim to replace the use of arbitrary factors of safety with explicit quantification of probabilities of success/failure. In the context of aircraft design, performance predictions take the form of aerodynamic databases that contain the expected forces and moments on an aircraft at all points in the flight envelope (defined by variables such as Mach number and altitude) and as a function of various parameters including aircraft orientation and the settings of multiple control surfaces. Uncertainties in these predictions can be used to create probabilistic aerodynamic databases that assign a probability distribution to each of these force and moment predictions. These probabilistic databases can then be sampled and used as an input to a flight simulator to determine the probability that a new aircraft design will meet certain performance or certification requirements \cite{wendorff_combining_2016}. Designing to maximize the probability of success provides a more robust design optimization framework \cite{ng_multifidelity_2014,multif} and supplies additional information to the engineer, which can be used to make better design decisions.

During the course of the typical aerospace design process, different kinds of performance analysis tools are used at different stages. Lower-fidelity computer simulations trade lower accuracy for faster computations and are useful at the very early stages of the design process, when the geometry of the aircraft is not well defined and is subject to significant change. They are often replaced with higher-fidelity simulations as the design progresses and more details of the design are finalized. Experimental data, normally obtained through a costly wind-tunnel test, typically provides the most accurate representation of the phenomena analyzed and is obtained late in the design process. Instead of discarding the low-fidelity simulation data when higher-fidelity data is available, there exist methods to combine data from multiple fidelity levels to generate better surrogate models. Multi-fidelity Gaussian processes have been developed and used extensively in this regard \cite{kennedy_predicting_2000,le_gratiet_recursive_2014}. They exploit the principles of co-kriging to incorporate multiple data sources into a single, robust, interpolation tool. The term multi-fidelity Gaussian processes (or MF GPs) is used to refer to these models and is the main focus of this paper.  These multi-fidelity models can be further improved using gradient information and hybrid bridge functions \cite{han_improving_2013}. While similar bridge-functions are used in this work (referred to as biases in Section \ref{sec:mf_modeling}), gradient-enhanced co-kriging is not explored due to the absence of, and the difficulty in obtaining gradient data.

These previous efforts have ignored the uncertainties in the simulations that produce the data. The uncertainty they do quantify is the prediction variance resulting from the choice of surrogate model parameters. Non-deterministic kriging \cite{bae_nondeterministic_2019} does well to estimate the noise in a process based on the data samples being used. If the source of data is a computer simulation, which is often the case in aerodynamics, the samples are deterministic. These samples exhibit no variation but yet have some uncertainty associated with them, usually due to model inadequacy. Building on \cite{huang_sequential_2006}, \cite{wendorff_combining_2016} uses multi-fidelity GPs to incorporate Subject Matter Expert (SME) defined uncertainties, in addition to the prediction variance mentioned above, to create probabilistic aerodynamic databases. These databases could be used to perform reliability-based design optimizations \cite{gaul_modified_2014}, adaptive sampling of the design space \cite{picheny_noisy_2014}, or flight maneuver analysis \cite{wendorff_combining_2016}.


This paper addresses some of the shortcomings of the aforementioned methods and provides a new framework that improves upon noisy multi-fidelity modelling techniques and eliminates the need for SME-defined uncertainties for CFD RANS calculations. Section \ref{sec:methodology} delves into the methodology used. This section first outlines the multi-fidelity Gaussian Process framework that is used to combine data from multiple sources that have varying levels of uncertainty in their predictions. Then, the section explains the method used to quantify the uncertainty arising from one such data source: Reynolds-Averaged Navier-Stokes (RANS) Computational Fluid Dynamics (CFD) simulations. Section \ref{sec:results} presents the resulting multi-fidelity aerodynamic databases for a full aircraft configuration, the NASA Common Research Model, for which data and uncertainties are generated using three separate levels of fidelity: vortex lattice, RANS CFD, and wind-tunnel experimental data. This is followed by a brief discussion of the outcomes in Section \ref{sec:conclusions} and suggestions of the way in which the methods presented and their associated uncertainties can be most effectively used in aerospace engineering moving forward.