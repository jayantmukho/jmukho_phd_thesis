\section{Contributions} \label{conclusion_contributions}

This thesis established a framework to perform virtual flight testing of an aircraft early in the design process.
This was achieved through contributions in the fields of uncertainty quantification in RANS CFD simulations (Chapter \ref{chap:rans_uq}), multi-fidelity Gaussian Processes (Chapter \ref{chap:mf_gp}), probabilistic aerodynamics and controls databases (Chapter \ref{chap:aero_db}), and certification by analysis (Chapter \ref{chap:cba}).

The eigenspace perturbation methodology was used to quantify epistemic uncertainties injected by turbulence models into RANS CFD simulations.
The methodology, developed by Emory et al. \cite{emory2013modeling} and Mishra et al. \cite{iaccarino_eig_pert}, was implemented in SU2, an open-source CFD solver. 
Test cases featuring a wide variety of complex flow conditions validated the uncertainty estimates from the methodology against high-fidelity experimental data. 
Investigation of the relationship between the epistemic uncertainties and numerical discretization errors involved running two test cases: subsonic flow over a 2D airfoil and transonic flow over a 3D wing.
The eigenspace perturbation methodology was applied to the sequentially refined mesh families used for grid convergence studies \cite{american_society_of_mechanical_engineers_standard_2009}.
The resulting uncertainty estimates were compared to the numerical discretization errors.
This comparison yielded that, given sufficient discretization to capture the relevant flow features, the epistemic uncertainty estimate was independent of the level of mesh refinement.
This observation justified using coarser meshes for UQ purposes, thereby reducing the computational cost of the methodology.   
The UQ methodology was applied to two aircraft, the NASA Common Research Model and the Generic T-tail Transport, to create aerodynamic databases with physics-informed uncertainties. 

This work used multi-fidelity Gaussian processes (GP) to combine different information sources and their associated uncertainties.
The auto-regressive formulation by Gratiet \cite{gratiet_multi-fidelity_nodate} was extended to use noisy data when the design sets of successive fidelity levels are not nested (Equations \ref{equ:mu_Zt} and \ref{equ:sig_Zt}).
Corresponding extensions to the parameter estimation equations were also presented (Equation \ref{equ:param_est_mf}).
These multi-fidelity GP equations were used to represent the aerodynamic predictions, and their associated uncertainties, for the NASA CRM.
AVL simulations, RANS CFD simulations, and wind tunnel experiments informed the low-, medium-, and high-fidelity levels, respectively.
Comparisons between the multi-fidelity and high-fidelity aerodynamic databases for the NASA CRM demonstrated the benefits of using multi-fidelity data fusion, especially when high-fidelity data is sparse or localized to a limited region of the domain. 

The application of the multi-fidelity modeling was extended to the GTT aircraft.
Probabilistic, multi-fidelity, and multi-dimensional aerodynamics and controls databases representing a full-configuration aircraft's lateral and longitudinal dynamics were created.
Low- and medium-fidelity data was generated using AVL and SU2, respectively.
Existing data from wind tunnel experiments informed the highest fidelity.
Uncertainties for the low-fidelity AVL simulations and the high-fidelity wind tunnel experiments were estimated using prior knowledge of the tools and their shortcomings. 
The eigenspace perturbation methodology provided uncertainties for the medium-fidelity CFD simulations. 
A total of $29$ separate Gaussian processes were required to model the databases; a separate GP represented each quantity of interest and its associated uncertainties.

With all aspects of an aircraft's flight dynamics defined, The Boeing Company's $5$ degree of freedom flight simulator was used to perform a flight certification maneuver computationally. 
The roll capability maneuver from the FAA's flight testing guide was chosen for this analysis \cite{romanowski_flight_2018}.
With $1000$ deterministic samples of the databases, a Monte Carlo analysis propogated the uncertainties in the aircraft's design analyses through the maneuver simulation. 
The resulting cumulative distribution functions of the performance metrics for the maneuver were analyzed to explicitly quantify the likelihood of the aircraft succeeding or failing the maneuver.
This certification analysis was performed with databases using varying fidelity levels to simulate different stages in the aircraft design process.
Furthermore, design decisions to mitigate the failure rate to a specific value were suggested. 