\section{Databases for the Generic T-tail Transport Aircraft} \label{sec:gtt_dbs}

Thus far, the examples of probabilistic databases have focused on the baseline aerodynamics of the aircraft in question. 
This section will focus on controls databases that define the aircraft's response to a control surface deflection by providing the imparted rotational moments. 
As mentioned in Section \ref{subsec:gtt_cfd_data_gen}, the difficulty in modeling control surface deflections precludes the use of CFD simulation data in building these controls databases. 
Consequently, these databases can be, at most, two-fidelity ones with AVL simulations and wind tunnel experiments as the data sources. 
This section presents several visualizations of the GP regressions.
Single-fidelity GP results using AVL data and WT data in isolation are contrasted with the two-fidelity GP results that use both sets of data to showcase the advantages of using multi-fidelity data.

In the interest of brevity and since the maneuver of interest focuses on the roll-authority of the aircraft, these visualizations will present the roll moment imparted due to aileron deflection ($C_{l_{\delta_a}}$).
This variable is just one of the 18 coefficients (Table \ref{tab:control_db}) that define the aircraft's control behavior, but it is the most pivotal for the maneuver of interest. 

\subsection{Single-fidelity Databases}

Figures \ref{fig:gtt_avl_ctrl_gps} and \ref{fig:gtt_wt_ctrl_gps} present the results of the single-fidelity GP regression performed on AVL data and wind tunnel data, respectively. 
With controls databases, a new input variable defining the control surface deflection angle ($\delta_*$) is used in addition to $\alpha$ and $\beta$ .
This additional variable makes visualizing the results of the 3-dimensional GP in 3D space challenging. 
Both sets of figures show four surfaces or lines in the first three subfigures. 
Each line or surface corresponds to the GP prediction at a particular aileron deflection angle.
For each of the figures, the lines/surfaces are stacked in order of increasing deflection angle, where the angles are $\delta_a \in \{-25^\circ, -10^\circ, 10^\circ, 25^\circ\}$. 
These deflection angles are chosen because the wind tunnel experiments were carried out with these aileron deflection angles. 

\begin{figure}
    \centering
    \begin{subfigure}[\label{subfig:gtt_avl_ctrl_surf}3-Dimensional function in $\alpha$, $\beta$, and $\delta_a$] {
        \includegraphics[trim=0 0 0 0, clip, width=.48\textwidth]{code/image_gen/gmatt/1f/avl/images/gps/CRMAIL.png} }
    \end{subfigure}
    \hfill
    \begin{subfigure}[\label{subfig:gtt_avl_ctrl_beta}Variation in $\beta$ at $\alpha=8^\circ$]{
        \includegraphics[trim=0 0 0 0, clip, width=.48\textwidth]{code/image_gen/gmatt/1f/avl/images/gps/CRMAIL_alpha=8.png}
    }
    \end{subfigure}
    \hfill
    \begin{subfigure}[\label{subfig:gtt_avl_ctrl_alpha}Variation in $\alpha$ at $\beta=4^\circ$]{
        \includegraphics[trim=0 0 0 0, clip, width=.48\textwidth]{code/image_gen/gmatt/1f/avl/images/gps/CRMAIL_beta=4.png} 
    }
    \end{subfigure}
    \hfill
    \begin{subfigure}[\label{subfig:gtt_avl_ctrl_defl}Variation in $\delta_a$ at $\alpha = 8^\circ$ and $\beta=4^\circ$]{
        \includegraphics[trim=0 0 0 0, clip, width=.48\textwidth]{code/image_gen/gmatt/1f/avl/images/gps/CRMAIL_alpha=8_beta=4.png} 
    }
    \end{subfigure}
    \caption{Visualization of the 3-Dimensional AVL data and resulting single-fidelity GP for the rolling moment due to aileron deflections. \label{fig:gtt_avl_ctrl_gps}}
\end{figure}

\begin{figure}
    \centering
    \begin{subfigure}[\label{subfig:gtt_wt_ctrl_surf}3-Dimensional function in $\alpha$, $\beta$, and $\delta_a$] {
        \includegraphics[trim=0 0 0 0, clip, width=.48\textwidth]{code/image_gen/gmatt/1f/wt/images/gps/CRMAIL.png} }
    \end{subfigure}
    \hfill
    \begin{subfigure}[\label{subfig:gtt_wt_ctrl_beta}Variation in $\beta$ at $\alpha=8^\circ$]{
        \includegraphics[trim=0 0 0 0, clip, width=.48\textwidth]{code/image_gen/gmatt/1f/wt/images/gps/CRMAIL_alpha=8.png}
    }
    \end{subfigure}
    \hfill
    \begin{subfigure}[\label{subfig:gtt_wt_ctrl_alpha}Variation in $\alpha$ at $\beta=4^\circ$]{
        \includegraphics[trim=0 0 0 0, clip, width=.48\textwidth]{code/image_gen/gmatt/1f/wt/images/gps/CRMAIL_beta=4.png} 
    }
    \end{subfigure}
    \hfill
    \begin{subfigure}[\label{subfig:gtt_wt_ctrl_defl}Variation in $\delta_a$ at $\alpha = 8^\circ$ and $\beta=4^\circ$]{
        \includegraphics[trim=0 0 0 0, clip, width=.48\textwidth]{code/image_gen/gmatt/1f/wt_old/images/gps/CRMAIL_alpha=8_beta=4.png} 
    }
    \end{subfigure}
    \caption{Visualization of the 3-Dimensional WT data and resulting single-fidelity GP for the rolling moment due to aileron deflections. \label{fig:gtt_wt_ctrl_gps}}
\end{figure}

Figures \ref{subfig:gtt_avl_ctrl_surf} and \ref{subfig:gtt_wt_ctrl_surf} present stacks of surfaces where each surface represents $C_{l_{\delta_a}}$ as a function of $\alpha$ and $\beta$ at a particular aileron deflection angle ($\delta_a$).
Explicitly, the lowest surface in both figures corresponds to $\delta_a = -25^\circ$.
At the lowest level, the AVL data does capture the general trend of increasing $\delta_a$ leading to increased rolling moment, but there is a clear contrast in the linear trends learned from the AVL data and the non-linear trends represented in the wind tunnel data. 

In Figures \ref{subfig:gtt_avl_ctrl_beta} and \ref{subfig:gtt_wt_ctrl_beta}, the angle of attack is kept constant, $\alpha = 8^\circ$ and $C_{l_{\delta_a}}$ is represented as a function of $\beta$ at different values of $\delta_a$. 
For the AVL-based results (Figure \ref{subfig:gtt_avl_ctrl_beta}), blue circles and the associated error bars represent the uncertainty in the data.
The error bars are pretty large, and they result in overlapping of the gray $2\sigma$ uncertainty estimates at different $\delta_a$. 
For the WT-based results (Figure \ref{subfig:gtt_wt_ctrl_beta}), the uncertainty is a lot lower, but there is some discrepancy in the data points (black crosses) and the GP mean estimate (solid black line).
This discrepancy is not an error in the GP; instead, it is a consequence of the multi-dimensional hyper-parameters learned while maximizing the model's log-likelihood (Equation \ref{equ:hyp_param_sf}). 
Regularization prevents over-fitting of the data. 

For Figures \ref{subfig:gtt_avl_ctrl_alpha} and \ref{subfig:gtt_wt_ctrl_alpha}, the angle of slideslip is kept constant, $\beta=4^\circ$ and $C_{l_{\delta_a}}$ is represented as a function of $\alpha$ at different values of $\delta_a$.
There is a similar overlap of the uncertainty estimate for the AVL-based GP and a capturing of non-linear trends by the WT-based GP. 

Figures \ref{subfig:gtt_avl_ctrl_defl} and \ref{subfig:gtt_wt_ctrl_defl} present $C_{l_{\delta_a}}$ as a function of $\delta_a$ when $\alpha = 8^\circ$ and $\beta = 4^\circ$.
There is a distinctly linear relationship between the rolling moment and the aileron deflection. 
While the AVL-based GP respects the uncertainty in the data across the domain, the WT-based GP predicts increased uncertainty between data points. 
In these areas, the uncertainty in the model parameters is surpassing the uncertainty in the data.
This trend indicates the need for additional data between the current set of aileron deflection angles for which wind tunnel data is available.
Using the multi-fidelity framework addresses this need.

\subsection{Multi-fidelity Databases}
The lower-fidelity AVL data can augment the limited wind tunnel data points to create a better surrogate model.
Figure \ref{fig:gtt_2f_ctrl_gps} continues in the vein of the previous figures with stacks of lines and surfaces representing the rolling moment at different aileron deflection angles. 
When compared to the single-fidelity WT-based GP shown in Figure \ref{fig:gtt_wt_ctrl_gps}, the results are similar with the notable exception of Figure \ref{subfig:gtt_2f_ctrl_defl}.
When using the abundant AVL data to supplement the sparse WT data in the aileron deflection dimension, the ballooning of the uncertainty estimates between data points disappears. 
The uncertainty estimates across the domain are reduced without using any more high-fidelity data. 

\begin{figure}
    \centering
    \begin{subfigure}[\label{subfig:gtt_2f_ctrl_surf}3-Dimensional function in $\alpha$, $\beta$, and $\delta_a$] {
        \includegraphics[trim=0 0 0 0, clip, width=.48\textwidth]{code/image_gen/gmatt/3f/images/gps/CRMAIL.png} }
    \end{subfigure}
    \hfill
    \begin{subfigure}[\label{subfig:gtt_2f_ctrl_beta}Variation in $\beta$ at $\alpha=8^\circ$]{
        \includegraphics[trim=0 0 0 0, clip, width=.48\textwidth]{code/image_gen/gmatt/3f/images/gps/CRMAIL_alpha=8.png}
    }
    \end{subfigure}
    \hfill
    \begin{subfigure}[\label{subfig:gtt_2f_ctrl_alpha}Variation in $\alpha$ at $\beta=4^\circ$]{
        \includegraphics[trim=0 0 0 0, clip, width=.48\textwidth]{code/image_gen/gmatt/3f/images/gps/CRMAIL_beta=4.png} 
    }
    \end{subfigure}
    \hfill
    \begin{subfigure}[\label{subfig:gtt_2f_ctrl_defl}Variation in $\delta_a$ at $\alpha = 8^\circ$ and $\beta=4^\circ$]{
        \includegraphics[trim=0 0 0 0, clip, width=.48\textwidth]{code/image_gen/gmatt/3f/images/gps/CRMAIL_alpha=8_beta=4.png} 
    }
    \end{subfigure}
    \caption{Visualization of the 3-Dimensional AVL and WT data, and the resulting two-fidelity GP for the rolling moment due to aileron deflections. \label{fig:gtt_2f_ctrl_gps}}
\end{figure}

This feature is a significant advantage of using multi-fidelity data, but there is a caveat.
The lower fidelity data needs to capture some relevant trends in the high-fidelity data. 
If the lower-fidelity trends contradict those seen in the high-fidelity data, then the low-fidelity data only corrupts the GP predictions. 
The multi-fidelity GP is trained on the difference between the low-fidelity data and the high-fidelity data. 
If there is no correlation between the low-fidelity and the high-fidelity data, then the difference between the two sets of data provides no useful information and only adds noise to the system. 
This issue, in turn, would prevent the multi-fidelity fit from performing as well or better than the single-fidelity fits. 

With the numerous multi-fidelity GP defining the aerodynamics and controls databases for the GTT aircraft created, flight simulations can be run using deterministic samples taken from these models.
These samples introduce minor variations in the aircraft's performance metrics that respect the GP error estimates resulting from the uncertainty in the underlying analysis data.